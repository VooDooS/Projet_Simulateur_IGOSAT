\hypertarget{R_xC3_xA9f_xC3_xA9rence_classCreation}{}\section{Creation d'un nouveau classe}\label{R_xC3_xA9f_xC3_xA9rence_classCreation}
Pour créer un nouveau macro-\/module, il faut hériter une nouvelle classe de classe \hyperlink{classMacroModule}{Macro\-Module} qui va représenter votre nouveau macro-\/module. La classe \hyperlink{classMacroModule}{Macro\-Module} est héritée de classe \hyperlink{classModule}{Module}. Exemple de création du \hyperlink{classMacroModule}{Macro\-Module} {\bfseries \hyperlink{classBatteryModule}{Battery\-Module}}

{\ttfamily class \hyperlink{classBatteryController}{Battery\-Controller}\-:public \hyperlink{classModule}{Module}\{\};}

Il faut au moins redéfinir deux methodes
\begin{DoxyItemize}
\item {\ttfamily Constructeur(Params = Params())}

Declarez le constructeur de votre macro-\/module dans le fichier Macro\-Module\-Name.\-h où Macro\-Module\-Name est le nom de classe de votre macro-\/module. Le constructeur d'un nouveau module doit utiliser le constructeur de base de classe \hyperlink{classMacroModule}{Macro\-Module}\-:


\begin{DoxyCode}
\hyperlink{classBatteryModule_a2fb494ef5f124c38c0fdf9ccfb31918f}{BatteryModule::BatteryModule}(std::string name, 
      Params params) : \hyperlink{classMacroModule}{MacroModule}(name, params, \textcolor{stringliteral}{"
      BatteryModule/BatteryModule.xml"})\{
    \textcolor{comment}{/* ce qu'il faut faire pendant la création d'un module */}
\}
\end{DoxyCode}


Dans cet constructeur de base\-:
\begin{DoxyEnumerate}
\item \hyperlink{classBatteryModule}{Battery\-Module} – nom de votre macro-\/module.
\item params – un objet de type {\ttfamily std\-::unordered\-\_\-map$<$std\-::string, double$>$} avec des parametrès d'un macro-\/module.
\item \char`\"{}\-Battery\-Module/\-Battery\-Module.\-xml\char`\"{} – le chemin vers un fichier {\ttfamily .xml} qui contient la descripton des parametres d'un macro-\/module et la description des messages qui peuvent être traités par ce macro-\/module.
\end{DoxyEnumerate}
\end{DoxyItemize}\hypertarget{R_xC3_xA9f_xC3_xA9rence_properties}{}\section{Propriétés d'un Module}\label{R_xC3_xA9f_xC3_xA9rence_properties}
\hypertarget{R_xC3_xA9f_xC3_xA9rence_submodules}{}\subsection{Sous-\/modules}\label{R_xC3_xA9f_xC3_xA9rence_submodules}
La difference principale entre \hyperlink{classModule}{Module} et \hyperlink{classMacroModule}{Macro\-Module} est la possibilité d'ajouter sous-\/modules dans \hyperlink{classMacroModule}{Macro\-Module} et établir les liens entre eux. \hypertarget{R_xC3_xA9f_xC3_xA9rence_addsubmodule}{}\subsubsection{Ajout d'un sous-\/module}\label{R_xC3_xA9f_xC3_xA9rence_addsubmodule}
Ajout d'um module à macro-\/module peut être effectué par methode {\ttfamily add\-Sub\-Module(\-Module)}. Il faut le faire dans constructeur de macro-\/module. Exemple de constructeur \hyperlink{classBatteryModule}{Battery\-Module} avec ajout des sous-\/modules \hyperlink{classBattery}{Battery} et \hyperlink{classBatteryController}{Battery\-Controller}\-: 
\begin{DoxyCode}
\hyperlink{classBatteryModule_a2fb494ef5f124c38c0fdf9ccfb31918f}{BatteryModule::BatteryModule}(std::string name, 
      Params params) : \hyperlink{classMacroModule}{MacroModule}(name, params, \textcolor{stringliteral}{"
      BatteryModule/BatteryModule.xml"})\{
    \textcolor{comment}{//Les modules:}
    addSubModule(\textcolor{keyword}{new} \hyperlink{classBattery}{Battery}());
    addSubModule(\textcolor{keyword}{new} \hyperlink{classBatteryController}{BatteryController}());        \}
\end{DoxyCode}
\hypertarget{R_xC3_xA9f_xC3_xA9rence_connectSubModules}{}\subsubsection{Etablissement des connections entre sous-\/modules}\label{R_xC3_xA9f_xC3_xA9rence_connectSubModules}
Etablissement de connections entre sous-\/modules peut-\/être effectué par méthode {\ttfamily add\-Connexion(\-Connexion $\ast$)}. Exemple de constructeur \hyperlink{classBatteryModule}{Battery\-Module} avec etablissement des connections entre sous-\/modules \hyperlink{classBatteryController}{Battery\-Controller} (en utilisant socket \char`\"{}to\-Battery\-Controller\char`\"{}) et \hyperlink{classBattery}{Battery} (en utilisant socket \char`\"{}to\-Battery\char`\"{})\-: 
\begin{DoxyCode}
\hyperlink{classBatteryModule_a2fb494ef5f124c38c0fdf9ccfb31918f}{BatteryModule::BatteryModule}(std::string name, 
      Params params) : \hyperlink{classMacroModule}{MacroModule}(name, params, \textcolor{stringliteral}{"
      BatteryModule/BatteryModule.xml"})\{
     addConnexion(\textcolor{keyword}{new} \hyperlink{classConnexion}{Connexion}(getModuleByName(\textcolor{stringliteral}{"Battery"})->
      getSocketByName(\textcolor{stringliteral}{"toBatteryController"}), getModuleByName(\textcolor{stringliteral}{"BatteryController"})->
      getSocketByName(\textcolor{stringliteral}{"toBattery"})));
\}
\end{DoxyCode}
\hypertarget{R_xC3_xA9f_xC3_xA9rence_classCreation}{}\section{Creation d'un nouveau classe}\label{R_xC3_xA9f_xC3_xA9rence_classCreation}
Pour créer un nouveau module, il faut hériter un nouveau classe de classe \hyperlink{classModule}{Module} qui va représenter votre nouveau module. Exemple de création de \hyperlink{classModule}{Module} {\bfseries \hyperlink{classBatteryController}{Battery\-Controller}}

{\ttfamily class \hyperlink{classBatteryController}{Battery\-Controller}\-:public \hyperlink{classModule}{Module}\{\};}

Il faut au moins redéfinir deux methodes
\begin{DoxyItemize}
\item {\ttfamily Constructeur(Params = Params())}

Declarez le constructeur de votre module dans le fichier Module\-Name.\-h Le constructeur d'un nouveau module doit utiliser le constructeur de base de classe \hyperlink{classModule}{Module}\-:


\begin{DoxyCode}
BatteryController::BatteryController(Params params) : \hyperlink{classModule}{Module}(\textcolor{stringliteral}{"
      BatteryController"}, params, \textcolor{stringliteral}{"BatteryModule/BatteryController.xml"})\{
   \textcolor{comment}{/* ce qu'il faut faire pendant la création d'un module */}
\}
\end{DoxyCode}


Dans cet constructeur de base\-:
\begin{DoxyEnumerate}
\item \hyperlink{classBatteryController}{Battery\-Controller} – nom de votre module.
\item params – un objet de type {\ttfamily std\-::unordered\-\_\-map$<$std\-::string, double$>$} avec des parametrès d'un module.
\item \char`\"{}\-Battery\-Module/\-Battery\-Controller.\-xml\char`\"{} – le chemin vers un fichier {\ttfamily .xml} qui contient la descripton des parametres d'un module et la description des messages qui peuvent être traités par ce module.
\end{DoxyEnumerate}
\end{DoxyItemize}


\begin{DoxyItemize}
\item process(std\-::shared\-\_\-ptr$<$\-Message$>$)

C'est une méthode qui traite des messages qui peuvent être traités par ce module. La sélection d'un message peut être réalisé par déclaration {\ttfamily if} qui verifié le nom d'un message en utilisant la methode {\ttfamily get\-Name()} d'un objet \hyperlink{classMessage}{Message}. Par exemple, le traitement des messages \char`\"{}get\-Status\char`\"{} et \char`\"{}actual\-Voltage\char`\"{} de \hyperlink{classModule}{Module} \char`\"{}\-Battery\-Controller\char`\"{} peut être implementé d'une telle manière\-:


\begin{DoxyCode}
\textcolor{keywordtype}{void} BatteryController::process(std::shared\_ptr<Message> m)\{
    \textcolor{keywordflow}{if} (m->getName() == \textcolor{stringliteral}{"getStatus"}) \{
       \textcolor{comment}{//ce qu'il faut faire au reçu d'un message "getStatus"}
    \}

    \textcolor{keywordflow}{if} (m->getName() == \textcolor{stringliteral}{"actualVoltage"}) \{
        \textcolor{comment}{//ce qu'il faut faire au reçu d'un message "actualVoltage"}
    \}
\}
\end{DoxyCode}

\end{DoxyItemize}\hypertarget{R_xC3_xA9f_xC3_xA9rence_properties}{}\section{Propriétés d'un Module}\label{R_xC3_xA9f_xC3_xA9rence_properties}
\hypertarget{R_xC3_xA9f_xC3_xA9rence_memory}{}\subsection{Mémoire}\label{R_xC3_xA9f_xC3_xA9rence_memory}
\hypertarget{R_xC3_xA9f_xC3_xA9rence_parameters}{}\subsection{Paramètres}\label{R_xC3_xA9f_xC3_xA9rence_parameters}
Le parametre de module est une valeur quelconque de type float avec le nom correspondant. Il existe deux façons d'ajout de parametres au \hyperlink{classModule}{Module}
\begin{DoxyItemize}
\item Par fichier .xml (regardez la section correspondante)
\item Par parametre de construction d'un \hyperlink{classModule}{Module}
\end{DoxyItemize}


\begin{DoxyCode}
\textcolor{comment}{//Paramètres:}
unordered\_map<string, double> p;

p[\textcolor{stringliteral}{"voltage"}] = 40;
p[\textcolor{stringliteral}{"amperage"}] = 0.2;
p[\textcolor{stringliteral}{"capacity"}] = 10000;
p[\textcolor{stringliteral}{"TEMP1"}] = 30;
p[\textcolor{stringliteral}{"TEMP2"}] = 35;
p[\textcolor{stringliteral}{"TEMP3"}] = 40;
p[\textcolor{stringliteral}{"TEMP4"}] = 45;

battery = \textcolor{keyword}{new} \hyperlink{classBattery}{Battery}(p);
\end{DoxyCode}
\hypertarget{R_xC3_xA9f_xC3_xA9rence_sockets}{}\subsection{Sockets}\label{R_xC3_xA9f_xC3_xA9rence_sockets}
Le \hyperlink{classSocket}{Socket} est un point de communication qui permet de connecter des modules differents et effectuer l'echangement de messages entre eux. Il existe deux façons d'ajout de sockets au \hyperlink{classModule}{Module} \begin{DoxyItemize}
\item Par fichier .xml (regardez la section correspondante) \item Par utilisation de méthode add\-Socket dans le constructeur de classe\-: 
\begin{DoxyCode}
\hyperlink{classBatteryModule_a2fb494ef5f124c38c0fdf9ccfb31918f}{BatteryModule::BatteryModule}(std::string name, 
      Params params) : \hyperlink{classMacroModule}{MacroModule}(name, params)\{
    \textcolor{comment}{//Les connecteurs du macromodule:}
    addSocket(\hyperlink{classSocket}{Socket}(\textcolor{stringliteral}{"fromExt"}));
    addSocket(\hyperlink{classSocket}{Socket}(\textcolor{stringliteral}{"toScao"}));
\}
\end{DoxyCode}
\end{DoxyItemize}
\hypertarget{R_xC3_xA9f_xC3_xA9rence_messages}{}\subsection{Messages}\label{R_xC3_xA9f_xC3_xA9rence_messages}
Le \hyperlink{classMessage}{Message} est l'unité d'information qui peut être transferé entre des modules. Le \hyperlink{classMessage}{Message} peut être de type int, float ou string. Chaque message a son propre nom. Le \hyperlink{classModule}{Module} distingue des messages par leurs noms. Il faut definir préalablement definir une liste des messages qui peuvent être traités par un module. Il existe deux façons de le faire\-: \begin{DoxyItemize}
\item Par fichier .xml (regardez la section correspondante) \item Par utilisation de méthode {\ttfamily add\-Message (std\-::string, int)} dans le constructeur de classe \hyperlink{classModule}{Module}\-:\end{DoxyItemize}

\begin{DoxyCode}
BatteryController::BatteryController(Params params) : \hyperlink{classModule}{Module}(\textcolor{stringliteral}{"Battery
       Controller"}, params)\{
    \textcolor{comment}{//Les messages compris par le controlleur:}
    addMessage(\textcolor{stringliteral}{"getStatus"}, 5);
    addMessage(\textcolor{stringliteral}{"getStatus"}, 5);
\}
\end{DoxyCode}
 